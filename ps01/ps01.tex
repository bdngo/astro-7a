\documentclass{article}
\usepackage{amsmath, amssymb, amsfonts, amsthm}
\usepackage{cancel}
\usepackage[output-complex-root=j]{siunitx}
\usepackage[american, nooldvoltagedirection]{circuitikz}
\usepackage{bm}
\usepackage{listings}
\usepackage{graphicx}
\usepackage{fullpage}
\usepackage{hyperref}

\renewcommand{\thesection}{\arabic{section}}
\renewcommand{\thesubsection}{\thesection.\alph{subsection}}
\renewcommand{\thesubsubsection}{\thesubsection.\roman{subsubsection}}

\newtheorem{genthm}{Theorem}
\DeclareSIUnit\year{yr}

\newcommand{\unit}[1]{\bm{\hat{#1}}}
\newcommand{\iprod}[2]{\left\langle #1, #2 \right\rangle}
\newcommand{\tpose}[1]{\left[#1\right]^{\! \top} \!\!}
\newcommand{\diff}[1]{\frac{d}{d #1}}

\lstset{
    language=Python,
    tabsize=4,
    basicstyle=\ttfamily,
    numbers=left,
    numberstyle=\ttfamily,
    keywordstyle=\color{blue},
    frame=single
}

\title{Astro 7A PS01}
\author{Bryan Ngo}
\date{2020-08-28}

\begin{document}

\maketitle

\section{Julian Dates}

\subsection{}

The first step is converting 11:10 PDT into UTC.
Since PDT is UTC-07:00, lecture started 18:10 UTC.
Converting into Julian time given the reference date 2459087,
\begin{equation}
    2459087 + \underbrace{2}_{\text{whole days}} + \underbrace{\frac{1}{4}}_{\text{six hours}} + \underbrace{\frac{10}{1440}}_{\text{minutes per day}} = 2459089.2569\bar{4}
\end{equation}

\subsection{}

The Modified Julian Date is \(2459089.2569\bar{4} - 2400000.5 = 59088.7569\bar{4}\).

\section{Synodic \& Sidereal Periods}

\subsection{}

The \emph{synodic period} of a planet is the time it takes to return to the same location in its orbit relative to the planet's sun.
The \emph{sidereal period}, on the other hand, is the time it takes to return to the same location in its orbit relative to the background stars.

\subsection{}

Synodic periods are relevant because they can be useful in determining the time between oppositions and conjunctions of planets.

\subsection{}

\begin{equation}
    \frac{1}{S} =
    \begin{cases}
        \frac{1}{P} - \frac{1}{P_\oplus} & \text{inferior} \\
        \frac{1}{P_\oplus} - \frac{1}{P} & \text{superior}
    \end{cases}
\end{equation}
Finding the case for an superior orbit, let \(P\) is the sidereal period of a planet.
Then the angular velocity is \(\frac{360}{P} \, \si{\deg\per\year}\).
We can then define the sidereal period of Earth to be \(P_\oplus = \SI{1}{\year}\), and thus can define the orbital rate to be \(\frac{360}{P_\oplus}\, \si{\deg\per\year}\).
Since the planet is in a superior orbit, Earth will have sweeped out a remainder angle \(\theta\) by the time it "catches up" with the planet.
The orbital rate is then
\begin{align}
    \frac{360}{P_\oplus} &= \frac{360 + \theta}{S} \\
    &= \frac{360}{S} + \frac{\theta}{S}
\end{align}
We can determine what \(\frac{\theta}{S}\) is by observing that this rate is the same rate at which our superior planet orbits, that being \(\frac{360}{P}\).
Thus,
\begin{align}
    \frac{\cancel{360}}{P_\oplus} &= \frac{\cancel{360}}{S} + \frac{\cancel{360}}{P}
    \Rightarrow \frac{1}{S} = \frac{1}{P_\oplus} + \frac{1}{P}
\end{align}
For the inferior orbit case, we can change our perspective to that of the superior planet, so we only need to switch the position of \(P\) and \(P_\oplus\).

\subsection{}

\begin{align}
    P_e &= \SI{6.1}{\day} \\
    P_c &= \SI{2.4}{\day} \\
    P_f &= \SI{9.2}{\day} \\
\end{align}
Relative to \(e\), \(c\) is inferior given that \(P_c < P_e\), and \(f\) is superior given that \(P_f > P_e\).
The synodic periods are thus
\begin{align}
    S_c &= \frac{1}{\frac{1}{P_c} - \frac{1}{P_e}} \approx \SI{4.0}{\day} \\
    S_f &= \frac{1}{\frac{1}{P_e} - \frac{1}{P_f}} \approx \SI{18.1}{\day}
\end{align}

\section{Syene \& Alexandria}

\end{document}
