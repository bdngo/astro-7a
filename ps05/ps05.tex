\documentclass{article}
\usepackage{amsmath, amssymb, amsfonts, amsthm}
\usepackage{cancel}
\usepackage[output-complex-root=j]{siunitx}
\usepackage[american, nooldvoltagedirection]{circuitikz}
\usepackage{bm}
\usepackage{listings}
\usepackage{graphicx}
\usepackage{fullpage}
\usepackage{hyperref}

\renewcommand{\thesection}{\arabic{section}}
\renewcommand{\thesubsection}{\thesection.\alph{subsection}}
\renewcommand{\thesubsubsection}{\thesubsection.\roman{subsubsection}}
\newcommand{\lemmaautorefname}{Lemma}

\newtheorem{theorem}{Theorem}
\newtheorem{lemma}{Lemma}
\DeclareSIUnit\parsec{pc}

\newcommand{\N}{\mathbb{N}}
\newcommand{\Z}{\mathbb{Z}}
\newcommand{\Q}{\mathbb{Q}}
\newcommand{\R}{\mathbb{R}}
\renewcommand{\C}{\mathbb{C}}
\newcommand{\unit}[1]{\bm{\hat{#1}}}
\newcommand{\iprod}[2]{\left\langle #1, #2 \right\rangle}
\newcommand{\tpose}[1]{\left[#1\right]^{\! \top} \!\!}
\newcommand{\diff}[1]{\frac{d}{d #1}}

\lstset{
    language=Python,
    tabsize=4,
    basicstyle=\ttfamily,
    numbers=left,
    numberstyle=\ttfamily,
    keywordstyle=\color{blue},
    frame=single
}

\title{ASTRO 7A PS 05}
\author{Bryan Ngo}
\date{2020-10-10}

\begin{document}

\maketitle

\section{Virtual Field Trip}

\subsection{}

\begin{align}
    D_L &= \SI{0.9144}{\meter} \\
    D_Y &= \SI{1.016}{\meter} \\
    A_L &= \pi \frac{D_L^2}{4} = \SI{0.66}{\meter\squared} \\
    A_Y &= \pi \frac{D_Y^2}{4} = \SI{0.81}{\meter\squared} \\
    \theta_L &= 1.22 \frac{\SI{600}{\nano\meter}}{D_L} = \SI{8.0e-7}{\radian} = \SI{0.16}{\arcsecond} \\
    \theta_Y &= 1.22 \frac{\SI{600}{\nano\meter}}{D_Y} = \SI{7.2e-7}{\radian} = \SI{0.15}{\arcsecond}
\end{align}
The Yerkes refractor has a higher resolution since it can resolve objects with a smaller angle between them.

\subsection{}

\begin{align}
    f_L &= \SI{17.37}{\meter} \\
    f_Y &= \SI{19.3}{\meter} \\
    F_L &= \frac{f_L}{D_L} = \num{19} \\
    F_Y &= \frac{f_Y}{D_Y} = \num{19}
\end{align}
Both telescopes have the same focal ratio, so they are equally "fast".
By "fast", we mean how long the telescope must be exposed to light from the object to properly resolve an image.
A smaller focal ratio means a faster optical system since the collecting area is larger, and vice versa.

\subsection{}

\begin{equation}
    m = \frac{f_{obj}}{f_{eye}}
\end{equation}
Since we want to magnify the image of Saturn by a factor of \(50\) our \(m\)-factor increases by a factor of \(50\).
Thus, we need a focal length of the eyepiece to be divided by \(50\), or \SI{0.3474}{\meter}.

\subsection{}

\begin{equation}
    \begin{array}{c|c|c}
        1.22 \frac{\lambda}{D} & D_S = \SI{3.048}{\meter} & D_K = \SI{0.762}{\meter} \\
        \hline
        \SI{500}{\nano\meter} & \SI{2.0e-7}{\radian} = \SI{0.041}{\arcsecond} & \SI{8.0e-7}{\radian} = \SI{0.16}{\arcsecond} \\
        \hline
        \SI{1}{\micro\meter} & \SI{4.0e-7}{\radian} = \SI{0.08}{\arcsecond} & \SI{1.6e-6}{\radian} = \SI{0.33}{\arcsecond}
    \end{array}
\end{equation}

\subsection{}

It is not possible to obtain diffraction-limited observations using any of the telescopes since the seeing angle is greater than any of the diffraction limits, so they are resolvable using conventional methods.
Diffraction-limited observations mean they are no more than the diffraction limit.

\subsection{}

First, a reference beacon is shone into the sky to create a "star".
Then, the incoming light is taken and collimated.
This is bounced off the reflecting mirror and beam split into one beam for the correction program and another to the observatory instruments.
The correction program is a closed-feedback loop that analyzes the incoming light and deforms the mirror using small mechanical systems to make the incoming wave flat, cancelling out atmospheric distortion.

\subsection{}

\begin{enumerate}
    \item In the 1990s, the Lick Observatory created the first adaptive optics system to use a guide laser.
    \item Lick Observatory operates the fully robotic Automated Planet Finder for exoplanet detection.
    \item Using the Shane Telescope, Lick astronomers perfected the radial-velocity technique of exoplanet detection.
\end{enumerate}

\section{Eclipsing Binary}

\subsection{}

First, we find the radial velocities of each star, which can be derived from the formula
\begin{equation}
    \frac{\Delta \lambda}{\lambda_0} = \frac{v_r}{c}
\end{equation}
where \(\lambda_0 = \SI{656.281}{\nano\meter}\).
Plugging in,
\begin{align}
    v_{r1} &= c \frac{\SI{0.078}{\nano\meter}}{\lambda_0} = \SI{3.56e+4}{\meter\per\second} \\
    v_{r2} &= c \frac{\SI{0.032}{\nano\meter}}{\lambda_0} = \SI{1.46e+4}{\meter\per\second} \\
    v_r &= v_{r1} + v_{r2} = \SI{5.02e+4}{\meter\per\second}
\end{align}
Then, we can use the formulae
\begin{align}
    R_s &= \frac{v_r}{2} (t_a - t_b) = \SI{2.1e+8}{\meter} = \num{0.29} R_\odot \\
    R_\ell &= R_s + \frac{v_r}{2} (t_c - t_b) = \SI{7.7e+8}{\meter} = \num{1.1} R_\odot 
\end{align}

\subsection{}

The mass ratio is
\begin{equation}
    \frac{m_1}{m_2} = \frac{v_{r2}}{v_{r1}} = \num{0.41}
\end{equation}
Then, we can find the total mass using the equation
\begin{equation}
    m_1 + m_2 = \frac{P}{2 \pi G} \left(\frac{v_{r1} + v_{r2}}{\sin(i)}\right)^3 = \SI{2.17e+30}{\kilogram}
\end{equation}
where \(P = \SI{83.2}{\day}\) because that is how long it takes for the eclipsing, and \(i = \ang{90}\).
Finally,
\begin{align}
    m_1 &= \frac{m_1 + m_2}{1 + \frac{m_2}{m_1}} = \SI{6.32e+29}{\kilogram} = 0.32 M_\odot \\
    m_2 &= \frac{m_1 + m_2}{1 + \frac{m_1}{m_2}} = \SI{1.54e+30}{\kilogram} = 0.77 M_\odot
\end{align}

\subsection{}

Finding the flux of both stars, the primary, and secondary eclipse with respect to the absolute magnitude of the sun,
\begin{align}
    \frac{B_p}{B_0} &= 100^{\frac{m_0 - m_p}{5}} = \num{0.36} \\
    \frac{B_s}{B_0} &= 100^{\frac{m_0 - m_s}{5}} = \num{0.86}
\end{align}
Then, we use the equation
\begin{equation}
    \frac{1 - \frac{B_p}{B_0}}{1 - \frac{B_s}{B_0}} = \frac{T_1^4}{T_2^4} = \num{4.65} \implies \frac{T_1}{T_2} = \num{1.47}
\end{equation}
The temperatures of the stars are relatively similar to each other, with only a \(47\%\) difference.

\subsection{}

The distance of the system from Earth is \(d = \frac{1}{p} = \SI{9.3}{\parsec}\).
Then, finding the semimajor axis of the system,
\begin{equation}
    a = \left(\frac{P^2 G (m_1 + m_2)}{4 \pi^2}\right)^{\frac{1}{3}} = \SI{5.7e+10}{\meter}
\end{equation}
Then, the angular separation is \(\alpha = \frac{a}{d} = \SI{2.0e-7}{\radian}\).

\subsection{}

Finding the diffraction limit of the Keck telescopes using \(\lambda_0 = \SI{656.281}{\nano\meter}\),
\begin{equation}
    \theta = \frac{\lambda_0}{\SI{10}{\meter}} = \SI{8.0e-8}{\radian} < \SI{2.0e-7}{\radian}
\end{equation}
So the system is indeed resolvable with the Keck telescopes, barring atmospheric distortion.

\subsection{}

Then, using the distance modulus,
\begin{equation}
    M = m - 5 \log\left(\frac{d}{\SI{10}{\parsec}}\right) = \num{5.75}
\end{equation}

\subsection{}

The hotter star is the first one, since \(\frac{T_1}{T_2} > 1\).
First, we find the uneclipsed and cooler stars' flux by comparing it to the Sun,
\begin{align}
    L_{tot} &= L_\odot 100^{\frac{M_\odot - M}{5}} = \SI{1.5e+26}{\watt} \\
    L_{tot} &= 4 \pi \sigma (R_s^2 T_s^4 + R_\ell^2 T_\ell^4) = 4 \pi \sigma \left(R_s^2 T_s^4 + R_\ell^2 \left(\frac{T_\ell}{T_s} T_s\right)^4\right) \\
    \Rightarrow T_s^4 &= \frac{L_{tot}}{4 \pi \sigma \left(R_s^2 + R_\ell^2 \frac{T_\ell^4}{T_s^4}\right)} = \SI{5932}{\kelvin}
\end{align}
where \(M = 5.7\) as found above.

\section{The X-Ray Binary SMC X-1}

\subsection{}

Suppose we use the shifting in periods to measure the radial velocity,
\begin{equation}
    \frac{\Delta P}{P_0} = \frac{v_r}{c}
\end{equation}
where \(P_0 = \SI{0.714890}{\second}\) and \(\Delta P = \SI{750}{\micro\second}\).
The radial velocity is thus \SI{3.14e+5}{\meter\per\second}.
Since the mass function sets a lower limit on \(m_2\), we can simply find it by
\begin{equation}
    m_2 > \frac{P}{2 \pi G} v_r^3 = \SI{2.5e+31}{\kilogram}
\end{equation}

\subsection{}

Now that we have both radial velocities, the ratio of the masses is
\begin{equation}
    \frac{m_1}{m_2} = \frac{v_r}{K_C} = \num{15.7}
\end{equation}
This is all we can gather about the masses since we do not have any information about the inclination of the system.

\subsection{}

Since we have inclination, there are no unknowns for finding the total mass,
\begin{equation}
    m_1 + m_2 = \frac{P}{2 \pi G} (v_r + K_C)^3 = \SI{3.0e+31}{\kilogram} \\
\end{equation}
And we can also find the individual masses,
\begin{align}
    m_1 &= \frac{m_1 + m_2}{1 + \frac{m_2}{m_1}} = \SI{2.8e+31}{\kilogram} = 0.32 M_\odot \\
    m_2 &= \frac{m_1 + m_2}{1 + \frac{m_1}{m_2}} = \SI{1.8e+30}{\kilogram} = 0.77 M_\odot
\end{align}

\subsection{}

Judging by the graph, the eclipse takes up \(20\%\) of the total orbital period, or \SI{0.77}{\day}.
The radii are thus
\begin{align}
    v_r &= \SI{3.34e+5}{\meter\per\second} \\
    R_s &= \frac{v_r}{2} (t_a - t_b) = \SI{1.1e+10}{\meter} = \num{16} R_\odot
\end{align}

\end{document}
