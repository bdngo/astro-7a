\documentclass{article}
\usepackage{amsmath, amssymb, amsfonts, amsthm}
\usepackage{cancel}
\usepackage[output-complex-root=j]{siunitx}
\usepackage[american, nooldvoltagedirection]{circuitikz}
\usepackage{bm}
\usepackage{listings}
\usepackage{graphicx}
\usepackage{pgfplots}
\usepackage{fullpage}
\usepackage{hyperref}

\renewcommand{\thesection}{\arabic{section}}
\renewcommand{\thesubsection}{\thesection.\alph{subsection}}
\renewcommand{\thesubsubsection}{\thesubsection.\roman{subsubsection}}

\newtheorem{theorem}{Theorem}
\DeclareSIUnit\year{yr}
\DeclareSIUnit\parsec{pc}
\DeclareSIUnit\erg{erg}

\newcommand{\N}{\mathbb{N}}
\newcommand{\Z}{\mathbb{Z}}
\newcommand{\Q}{\mathbb{Q}}
\newcommand{\R}{\mathbb{R}}
\renewcommand{\C}{\mathbb{C}}
\newcommand{\unit}[1]{\bm{\hat{#1}}}
\newcommand{\iprod}[2]{\left\langle #1, #2 \right\rangle}
\newcommand{\tpose}[1]{\left[#1\right]^{\! \top} \!\!}
\newcommand{\diff}[1]{\frac{d}{d #1}}

\lstset{
    language=Python,
    tabsize=4,
    basicstyle=\ttfamily,
    numbers=left,
    numberstyle=\ttfamily,
    keywordstyle=\color{blue},
    frame=single
}

\title{ASTRO 7A PS03}
\author{Bryan Ngo}
\date{2020-09-14}

\begin{document}

\maketitle

\section{Globular Cluster}

The sum of the kinetic energy of the globular cluster is
\begin{equation}
    \langle K \rangle = \sum_i \frac{1}{2} m_i \bar{v}^2 = \frac{1}{2} M \bar{v}^2
\end{equation}
where \(\bar{v} = \SI{20}{\kilo\meter\per\second}\) is the average velocity of a particle in the system.
The potential energy of the system is
\begin{equation}
    \langle U \rangle = -\frac{G M^2}{R}
\end{equation}
where \(R = \SI{1}{\parsec} = \SI{3.0857e+16}{\meter}\) is the radius of the system.
By the virial theorem,
\begin{align}
    E &= \frac{1}{2} \langle U \rangle \\
    \langle K \rangle + \langle U \rangle &= \frac{1}{2} \langle U \rangle \\
    \Rightarrow \langle K \rangle &= -\frac{1}{2} \langle U \rangle \\
    \frac{1}{2} M \bar{v}^2 &= \frac{1}{4} \frac{G M^2}{R} \\
    \Rightarrow M &= \frac{2 R \bar{v}^2}{G} = \SI{3.70e+35}{\kilogram} = \num{1.86e+5} M_\odot
\end{align}

\section{Multiples, Clusters, \& Magnitudes}

\subsection{}

We can derive an equation for converting the negative logarithmic scale of apparent magnitude to linear brightness.
This equation is
\begin{equation}
    b = 100^{-\frac{m}{5}}
\end{equation}
We can verify this by knowing that a star of \(m_1 = 0\) will be \(F_2\) times brighter than a star with \(m_2\).
With this in mind, our brightness scale is linear so we can add them,
\begin{align}
    100^{-\frac{m_{12}}{5}} &= 100^{-\frac{m_1}{5}} + 100^{-\frac{m_2}{5}} = \num{8.82e+2} \\
    \Rightarrow m_{12} &= -2.5 \log(\num{8.82e+2}) = \num{2.64}
\end{align}

\subsection{}

We can use the equation
\begin{equation} \label{eq:2b}
    \frac{L}{L_\odot} = 100^{\frac{M_\odot - M}{5}}
\end{equation}
where \(M_\odot = \num{4.83}\) and \(L_\odot = \SI{3.828}{\watt}\) in order to solve for the luminosity ratio of each star.
With that in mind, we take the linear combination of their luminosity ratios then use the equation in reverse to solve for the new absolute magnitude:
\begin{align}
    \frac{L_1}{L_\odot} &= \num{85.5} \\
    \frac{L_2}{L_\odot} &= \num{34.0} \\
    \frac{L_3}{L_\odot} &= \num{5.40} \\
    \frac{L_4}{L_\odot} &= \num{0.855} \\
    \Rightarrow \frac{50 L_1 + 200 L_2 + 800 L_3 + 6000 L_4}{L_\odot} &= 100^{\frac{M_\odot - M_{tot}}{5}} \\
    \Rightarrow M &= M_\odot - 2.5 \log\left(\frac{50 L_1 + 200 L_2 + 800 L_3 + 6000 L_4}{L_\odot}\right) = \num{-5.95}
\end{align}

\subsection{}

Using the distance modulus with \(d = \SI{3}{\kilo\parsec}\) and \(M = \num{-5.95}\),
\begin{align}
    m - M &= 5 \log\left(\frac{d}{\SI{10}{\parsec}}\right) \\
    \Rightarrow m &= M - 5 \log\left(300\right) = \num{-18.3}
\end{align}

\section{A Dazzling Double}

\subsection{}

Using the flux ratio equation,
\begin{equation}
    \frac{F_A}{F_B} = 100^{\frac{m_B - m_A}{5}} = \num{6.31}
\end{equation}

\subsection{}

Since they are orbiting each other, we can assume that their relative distances are very small compared to their absolute distances to Earth.
With that in mind, we can make the approximation that
\begin{equation}
    \frac{L_A}{L_B} \approx \frac{F_A}{F_B} = \num{6.31}
\end{equation}

\subsection{}

We can use combine our flux ratio with the true luminosity approximation to arrive at
\begin{align}
    \frac{F_A}{F_B} &= \frac{\frac{L_A}{\cancel{4 \pi} d_A^2}}{\frac{L_B}{\cancel{4 \pi} d_B^2}} = \frac{L_A d_B^2}{L_B d_A^2} \\
    \Rightarrow \left(\frac{d_B}{d_A}\right)^2 &= \frac{F_A L_B}{F_B L_A} \\
    \Rightarrow \frac{d_A}{d_B} &= \sqrt{\frac{F_B L_A}{F_A L_B}} = \num{0.91}
\end{align}
This means that Albireo B is more distant than Albireo A by the above factor.

\subsection{}

Using \autoref{eq:2b}, and replacing \(M_\odot\) and \(\frac{L}{L_\odot}\) with \(M_A = \num{-2.45}\) and the reciprocal of the previous luminosity ratio, respectively,
\begin{align}
    \frac{L_B}{L_A} &= 100^{\frac{M_A - M_B}{5}} \\
    \Rightarrow M_B &= M_A - 2.5 \log\left(\frac{L_B}{L_A}\right) = M_A + 2.5 \log\left(\frac{L_A}{L_B}\right) = \num{-0.66}
\end{align}

\section{Binning in Frequency vs. Wavelength}

\subsection{}

The equation describing the flux collected is
\begin{equation}
    F_\lambda = \frac{L}{A} = \frac{E}{A t}
\end{equation}
Since all the terms on the right hand side are known constants and \(\Delta \lambda = \SI{900}{\angstrom}\), the histogram should appear flat.
\begin{center}
    \begin{tikzpicture}
        \begin{axis}[
            title=Flux Density in Wavelength Space,
            width=\textwidth,
            height=\axisdefaultheight,
            ybar interval,
            xticklabel=\pgfmathprintnumber\tick--\pgfmathprintnumber\nexttick,
            tick label style={font=\footnotesize},
            xlabel={\(\lambda\) [\si{\angstrom}]},
            ylabel={\(F_\lambda\) [\si{\erg\per\centi\meter\squared}]},
            ymin=0, ymax=10e-5,
        ]
        \addplot+table[hist={bins=10}, x=wl, y=f, col sep=comma] {3a.csv};
        \end{axis}
    \end{tikzpicture}
\end{center}

\subsection{}

We can derive a similar equation to above,
\begin{equation}
    F_\lambda = \frac{L}{A} = \frac{E}{A t} = \frac{h \Delta \nu}{A t}
\end{equation}
where \(h = \SI{6.62607015e-27}{\erg\second}\) is the Planck constant.
The width of divisions is
\begin{equation}
    \Delta \nu = \frac{1}{N} \left(\frac{c}{\lambda_min} - \frac{c}{\lambda_{max}}\right) = \SI{2.70e+14}{\hertz}
\end{equation}
\begin{center}
    \begin{tikzpicture}
        \begin{axis}[
            title=Flux Density in Frequency Space,
            width=\textwidth,
            height=\axisdefaultheight,
            ybar interval,
            xticklabel=\pgfmathprintnumber\tick--\pgfmathprintnumber\nexttick,
            tick label style={font=\footnotesize},
            xlabel={\(\nu\) [\si{\hertz}]},
            ylabel={\(F_\nu\) [\si{\erg\per\centi\meter\squared}]},
            ymin=0, ymax=2e-26,
        ]
        \addplot+table[hist={bins=10}, x=wl, y=f, col sep=comma] {3b.csv};
        \end{axis}
    \end{tikzpicture}
\end{center}

\end{document}
