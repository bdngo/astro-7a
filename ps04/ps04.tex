\documentclass{article}
\usepackage{amsmath, amssymb, amsfonts, amsthm}
\usepackage{cancel}
\usepackage[output-complex-root=j]{siunitx}
\usepackage[american, nooldvoltagedirection]{circuitikz}
\usepackage{bm}
\usepackage{listings}
\usepackage{graphicx}
\usepackage{fullpage}
\usepackage{hyperref}

\renewcommand{\thesection}{\arabic{section}}
\renewcommand{\thesubsection}{\thesection.\alph{subsection}}
\renewcommand{\thesubsubsection}{\thesubsection.\roman{subsubsection}}

\newtheorem{theorem}{Theorem}

\newcommand{\N}{\mathbb{N}}
\newcommand{\Z}{\mathbb{Z}}
\newcommand{\Q}{\mathbb{Q}}
\newcommand{\R}{\mathbb{R}}
\renewcommand{\C}{\mathbb{C}}
\newcommand{\unit}[1]{\bm{\hat{#1}}}
\newcommand{\iprod}[2]{\left\langle #1, #2 \right\rangle}
\newcommand{\tpose}[1]{\left[#1\right]^{\! \top} \!\!}
\newcommand{\diff}[1]{\frac{d}{d #1}}

\lstset{
    language=Python,
    tabsize=4,
    basicstyle=\ttfamily,
    numbers=left,
    numberstyle=\ttfamily,
    keywordstyle=\color{blue},
    frame=single
}

\title{ASTRO 7A PS04}
\author{Bryan Ngo}
\date{2020-09-25}

\begin{document}

\maketitle

\section{Hot Dog}

\subsection{}

Using the Stefan-Boltzmann equation with \(A = \SI{0.75}{\meter\squared}\) and \(T = \SI{310}{\kelvin}\),
\begin{equation}
    L = A \sigma T^4 \implies L = \SI{392.8}{\watt}
\end{equation}

\subsection{}

Using Wien's displacement law,
\begin{equation}
    \lambda_{max} = \frac{\SI{2.897785}{\kelvin\meter}}{\SI{310}{\kelvin}} = \SI{9.35}{\micro\meter}
\end{equation}
which lies in the infrared region of the EM specturm.

\subsection{}

With \(T = \SI{295}{\kelvin}\),
\begin{equation}
    L = A \sigma T^4 \implies L = \SI{322.1}{\watt}
\end{equation}

\subsection{}

The total energy lost is
\begin{equation}
    L_{out} - L_{in} = \SI{70.7}{\watt}
\end{equation}

\section{Distant Relatives}

\subsection{}

The equation determining the total flux recieved from the KBO is
\begin{equation}
    F = \frac{L_\odot}{4 \pi R^2}
\end{equation}
Thus, the flux is
\begin{equation} \label{eq:3a}
    F = \frac{\SI{3.828e+26}{\watt}}{4 \pi (\SI{7.5e+12}{\meter})^2} = \SI{0.544}{\watt\per\meter\squared}
\end{equation}
From here, the flux received by the earth is simply
\begin{equation}
    F_\oplus = 0.15 \frac{\cancel{4 \pi} r^2 F}{ \cancel{4 \pi} D^2} = 0.15F \frac{r^2}{D^2} = \SI{9.73e-18}{\watt\per\meter\squared}
\end{equation}
where \(r, D\) is the radius of the KBO and the distance from the KBO to Earth (\SI{49}{\astronomicalunit}).

\subsection{}

Using the Stefan-Boltzmann equation,
\begin{equation}
    0.15F = \sigma T^4
\end{equation}
where \(F\) is as calculated in \autoref{eq:3a}.
Therefore,
\begin{equation}
    T = \sqrt[4]{\frac{0.15F}{\sigma}} = \SI{1200}{\kelvin}
\end{equation}

\subsection{}

\end{document}
