\documentclass{article}
\usepackage{amsmath, amssymb, amsfonts, amsthm}
\usepackage{cancel}
\usepackage[output-complex-root=j]{siunitx}
\usepackage[american, nooldvoltagedirection]{circuitikz}
\usepackage{bm}
\usepackage{listings}
\usepackage{graphicx}
\usepackage{fullpage}
\usepackage{hyperref}

\renewcommand{\thesection}{\arabic{section}}
\renewcommand{\thesubsection}{\thesection.\alph{subsection}}
\renewcommand{\thesubsubsection}{\thesubsection.\roman{subsubsection}}

\newtheorem{theorem}{Theorem}

\newcommand{\N}{\mathbb{N}}
\newcommand{\Z}{\mathbb{Z}}
\newcommand{\Q}{\mathbb{Q}}
\newcommand{\R}{\mathbb{R}}
\renewcommand{\C}{\mathbb{C}}
\newcommand{\unit}[1]{\bm{\hat{#1}}}
\newcommand{\iprod}[2]{\left\langle #1, #2 \right\rangle}
\newcommand{\tpose}[1]{\left[#1\right]^{\! \top} \!\!}
\newcommand{\diff}[1]{\frac{d}{d #1}}

\lstset{
    language=Python,
    tabsize=4,
    basicstyle=\ttfamily,
    numbers=left,
    numberstyle=\ttfamily,
    keywordstyle=\color{blue},
    frame=single
}

\title{ASTRO 7A PS04}
\author{Bryan Ngo}
\date{2020-09-25}

\begin{document}

\maketitle

\section{Hot Dog}

\subsection{}

Using the Stefan-Boltzmann equation with \(A = \SI{0.75}{\meter\squared}\) and \(T = \SI{310}{\kelvin}\),
\begin{equation}
    L = A \sigma T^4 \implies L = \SI{392.8}{\watt}
\end{equation}

\subsection{}

Using Wien's displacement law,
\begin{equation}
    \lambda_{max} = \frac{\SI{2.897785}{\kelvin\meter}}{\SI{310}{\kelvin}} = \SI{9.35}{\micro\meter}
\end{equation}
which lies in the infrared region of the EM specturm.

\subsection{}

With \(T = \SI{295}{\kelvin}\),
\begin{equation}
    L = A \sigma T^4 \implies L = \SI{322.1}{\watt}
\end{equation}

\subsection{}

The total energy lost is
\begin{equation}
    L_{out} - L_{in} = \SI{70.7}{\watt}
\end{equation}

\section{Distant Relatives}

\subsection{}

The equation determining the total flux recieved from the KBO is
\begin{equation}
    F = a \frac{L_\odot \cancel{\pi} r^2}{4 \cancel{\pi} R^2}
\end{equation}
where \(a, r, R\) is the albedo, radius of the KBO, and distance from the KBO (\SI{50}{\astronomicalunit}).
Thus, the received flux is
\begin{equation} \label{eq:3a}
    F = 0.15 \frac{(\SI{3.828e+26}{\watt}) (\SI{80}{\kilo\meter})^2}{4 (\SI{7.5e+12}{\meter})^2} = \SI{1.64e+9}{\watt}
\end{equation}
From here, the flux received by the earth is simply
\begin{equation}
    F_\oplus = \frac{F}{2 \pi (R - \SI{1}{\astronomicalunit})^2} = \SI{4.86e-18}{\watt}
\end{equation}
The factor of two in the numerator arises from the fact that only half of the surface area of the KBO is assumed to be emitting light.

\subsection{}

Using the Stefan-Boltzmann equation,
\begin{equation}
    F = (1 - a) \frac{L_\odot r^2}{4 R^2} = \SI{9.30e+9}{\watt}
\end{equation}
The \(1 - a\) term arises from the fact that since \(a\) proportion of the light is reflected, the rest must be absorbed as heat.
Therefore, the total flux radiated from the surface of the KBO is
\begin{equation}
    F_\oplus = \frac{F}{4 \pi r^2} = (1 - a) \frac{L_\odot}{16 \pi R^2} = \sigma T^4
\end{equation}
where the term is as normal because thermal radiation is assumed to radiate uniformly across the entire surface area.
Therefore,
\begin{equation}
    T = \sqrt[4]{\frac{F_\odot}{\sigma}} = \SI{37.8}{\kelvin}
\end{equation}

\subsection{}

Using Wien's displacement law, we find the peak wavelength to be
\begin{equation}
    \lambda_{max} = \frac{\SI{3e-3}{\meter\kelvin}}{T} = \SI{7.67e-5}{\meter} = \SI{7.67e+5}{\angstrom}
\end{equation}
Since the peak wavelength is far beyond the range of the V-band, this must mean the V-band lies at a very small wavelength relative to the blackbody.
Thus, we can use the Wien tail approximation,
\begin{eqnarray}
    B_\lambda(T) \approx \frac{2hc^2}{\lambda^5} e^{-\frac{hc}{\lambda k T}}
\end{eqnarray}
The combined thermal flux is simply
\begin{equation}
    I = \int_{\SI{5000}{\angstrom}}^{\SI{6000}{\angstrom}} \frac{2hc^2}{\lambda^5} e^{-\frac{hc}{\lambda k T}} \, d\lambda
\end{equation}
Letting \(u = \frac{hc}{\lambda k T}\) and \(du = -\frac{hc}{\lambda^2 k T} \, d\lambda\),
\begin{align}
    I &= \int_{\num{761.4}}^{\num{634.5}} \frac{2hc^2}{\lambda^3} \left(-\frac{k T}{h c}\right)e^{-u} \left(-\frac{hc}{\lambda^2 k T} \, d\lambda\right) \\
    &= -2ckT \int_{\num{761.4}}^{\num{634.5}} u^3 \left(\frac{kT}{hc}\right)^3 e^{-u} \, du \\
    &= \frac{2(kT)^4}{h^3c^2} \int_{\num{634.5}}^{\num{761.4}} u^3 e^{-u} \, du
\end{align}

\end{document}
