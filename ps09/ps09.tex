\documentclass{article}
\usepackage{eecstex}
\usepackage{mhchem}

\DeclareSIUnit\year{yr}
\DeclareSIUnit\parsec{pc}

\title{ASTRO 7A PS 09}
\author{Bryan Ngo}
\date{2020-11-21}

\begin{document}

\maketitle

\section{The Mass of the Sun}

\subsection{}

The sun uses \SI{1.3e+44}{\joule} of energy due to nuclear fusion producing a luminosity of \SI{3.828e+26}{\watt}.
Using the mass-energy equivalence and assuming luminosity originates from fusion,
\begin{equation}
    L_\odot = \frac{dE}{dt} = \frac{d (m c^2)}{dt} = \frac{dm}{dt} c^2 \implies \frac{dm}{dt} = \frac{L_\odot}{c^2} = \SI{4.26e+9}{\kilogram\per\second} = \num{6.72e-14} M_\odot \, \si{\per\year}
\end{equation}

\subsection{}

The nuclear fusion mass loss rate is about twice as fast as the solar wind loss, with \(\frac{m_{nuc}}{m_{sw}} = \num{2.24}\).

\subsection{}

Assuming a main sequence length of \SI{e+10}{\year},
\begin{align}
    m_{nuc} &= \num{6.72e-4} M_{\odot} \\
    m_{sw} &= \num{3e-4} M_{\odot}
\end{align}
This only adds up to about a thousandth of the Sun's mass over its entire main sequence.

\section{The North American Nebula}

\begin{align}
    A_\lambda &= \num{1.1} \\
    t &= \SI{20}{\parsec} \\
    d &= \SI{700}{\parsec} \\
    M_V &= \num{-1.1}
\end{align}
\begin{enumerate}
    \item \(m_1 = M_V + 5\log\left(\frac{d}{\SI{10}{\parsec}}\right) = \num{8.1}\)
    \item \(m_2 = M_V + 5\log\left(\frac{d + t}{\SI{10}{\parsec}}\right) + A_\lambda = \num{9.3}\)
    \item \(d_{est} = (\SI{10}{\parsec}) 10^{\frac{m_2 - M_V}{5}} = \SI{1195}{\parsec}\) and \(d_{act} = \SI{720}{\parsec}\), so \(\sigma = \frac{|d_{act} - d_{est}|}{d_{act}} = \num{0.66} = \num{66}\%\)
\end{enumerate}

\section{H I Clouds}

\begin{align}
    \tau_{\ce{H}} &= \num{0.5} \\
    T &= \SI{100}{\kelvin} \\
    \Delta v &= \SI{10}{\kilo\meter\per\second} \\
    n_{\ce{H}} &= \SI{e+7}{\per\meter\cubed} \\
    \tau_{\ce{H}} &= \num{5.2e-23} \frac{N_{\ce{H}}}{T \Delta v}
\end{align}
\begin{align}
    N_{\ce{H}} &= \frac{\tau_{\ce{H}} T \Delta v}{\num{5.2e-23}} = \SI{9.6e+27}{\per\meter\squared} = n_{\ce{H}} t \implies t = \frac{N_{\ce{H}}}{n_{\ce{H}}} = \SI{9.6e+20}{\meter} = \SI{3.12e+4}{\parsec}
\end{align}

\section{Molecular Transitions}

\begin{align}
    E_r &= \frac{1}{2} I \omega^2 = \frac{L^2}{2I} \\
    L &= \hbar \sqrt{\ell (\ell + 1)}, \ell \in \N
\end{align}

\subsection{}

\begin{theorem}
    Given masses \(m_1, m_2\) and distances from their center of mass \(r_1, r_2\), respectively,
    \begin{equation}
        I = m_1 r_1^2 + m_2 r_2^2 = \mu r^2
    \end{equation}
    where \(\mu = \frac{m_1 m_2}{m_1 + m_2}\) and \(r = r_1 + r_2\).
\end{theorem}
\begin{proof}
    We can define \(r_1, r_2\) in terms of \(r\) as follows:
    \begin{align}
        r_1 &= \frac{\mu}{m_1} r \\
        r_2 &= \frac{\mu}{m_2} r
    \end{align}
    Plugging into the original equation for moment of inertia,
    \begin{align}
        I = m_1 r_1^2 + m_2 r_2^2 &= \cancel{m_1} \frac{\mu^2}{m_1^{\cancel{2}}} r^2 + \cancel{m_2} \frac{\mu^2}{m_2^{\cancel{2}}} r^2 \\
        &= \mu^2 r^2 \left(\frac{1}{m_1} + \frac{1}{m_2}\right) \\
        &= \mu^2 r^2 \underbrace{\left(\frac{m_1 + m_2}{m_1 m_2}\right)}_{\frac{1}{\mu}} \\
        &= \mu r^2
    \end{align}
\end{proof}

\subsection{}

\begin{align}
    m_{\ce{^{12}C}} &= \SI{12.000}{\atomicmassunit} \\
    m_{\ce{^{13}C}} &= \SI{13.003}{\atomicmassunit} \\
    m_{\ce{^{16}O}} &= \SI{15.995}{\atomicmassunit} \\
    r_{\ce{CO}} &= \SI{0.12}{\nano\meter} \\
    \mu_{\ce{^{12}CO}} &= \SI{6.86}{\atomicmassunit} = \SI{1.14e-26}{\kilogram} \\
    \mu_{\ce{^{13}CO}} &= \SI{7.17}{\atomicmassunit} = \SI{1.19e-26}{\kilogram} \\
    I_{\ce{^{12}CO}} &= \mu_{\ce{^{12}CO}} r_{\ce{CO}}^2 = \SI{1.64e-46}{\kilogram\meter\squared} \\
    I_{\ce{^{13}CO}} &= \mu_{\ce{^{13}CO}} r_{\ce{CO}}^2 = \SI{1.72e-46}{\kilogram\meter\squared}
\end{align}

\subsection{}

\begin{align}
    L_3 &= \SI{2.6e-34}{\joule\second} \\
    L_2 &= \SI{3.7e-34}{\joule\second} \\
    E_3 - E_2 &= \frac{L_3^2 - L_2^2}{2I_{\ce{^{12}CO}}} = \SI{2.03e-22}{\joule} \\
    E_{2 \to 3} &= \frac{h c}{\lambda} \implies \lambda = \frac{h c}{E_{2 \to 3}} = \SI{9.76e-4}{\meter} \\
\end{align}
which lies within the radio/infrared boundary.

\subsection{}

\begin{align}
    E_3 - E_2 = \frac{L_3^2 - L_2^2}{2I_{\ce{^{13}CO}}} &= \SI{1.94e-22}{\joule} \\
    E_{2 \to 3} &= \frac{h c}{\lambda} \implies \lambda = \frac{h c}{E_{2 \to 3}} = \SI{1.02e-3}{\meter}
\end{align}
which is in the radio range.
Astronomers can use the differences in wavelength and emitted spectrum to determine isotopes.

\section{Cooling Molecular Clouds \& Star Formation}

\subsection{}

\begin{align}
    M_c &= \num{1.5} M_\odot \\
    R &= \SI{0.1}{\parsec}
\end{align}

The gravitational acceleration \(g\) remains constant throughout the contraction, so
\begin{equation}
    g = \frac{G M_c}{R^2} = \frac{G \frac{4}{3} \pi R^3 \rho}{R^2} = \frac{4}{3} \pi \rho G R
\end{equation}
Then, using kinematic equations to solve for \(\frac{R}{2}\),
\begin{align}
    \frac{1}{2} g t^2 &= \frac{R}{2} \\
    \frac{2}{3} \pi \rho G R t^2 &= \frac{R}{2} \\
    \Rightarrow t &= \sqrt{\frac{3}{4 \pi G \rho}}
\end{align}

\subsection{}

\begin{align}
    r &= \SI{0.1}{\micro\meter} \\
    \rho_p &\approx \SI{1}{\gram\per\centi\meter\cubed} = \SI{1000}{\kilogram\per\meter\cubed} \\
    \rho_d &\approx Z \rho \\
    Z &\approx \num{e-2}
\end{align}
Using the mean collision time,
\begin{equation}
    t = \frac{\ell}{v} = \frac{1}{n \sigma v}
\end{equation}
where \(\ell\) is the mean free path.
Substituting the collisional cross-section \(\sigma\), number density \(n\), and rms velocity \(v\), 
\begin{align}
    t &= \left(\frac{\rho_d}{\frac{4}{3} \pi \rho_p r^3} \pi 4r^2 \sqrt{\frac{3kT}{\mu m_{\ce{H}}}}\right)^{-1} = \left(\frac{3 \rho_d}{\rho_p r}\ \sqrt{\frac{3kT}{\mu m_{\ce{H}}}}\right)^{-1} = \frac{\rho_p r}{3 \rho_d} \sqrt{\frac{\mu m_{\ce{H}}}{3kT}}
\end{align}
where we find \(n = \frac{\rho_d}{M_c}\), where \(M_c = \rho_p V\).

\subsection{}

The density of the cloud is \(\frac{M_c}{\frac{4}{3} \pi R^3} = \SI{2.4e-17}{\kilogram\per\meter\cubed}\), and \(\mu = 2\) since the cloud is mainly composed of diatomic hydrogen.
Using the Jeans radius to solve for \(T\),
\begin{equation}
    R = R_j = \left(\frac{15kT}{4 \pi G \mu m_{\ce{H}} \rho}\right)^{\frac{1}{2}} \implies T = \frac{4 \pi G \mu m_{\ce{H}} \rho R^2}{15k}
\end{equation}
Then, calculating \(t_{\text{ff}}\) and \(t_c\) by plugging in \(T\),
\begin{align}
    t_{\text{ff}} &= \sqrt{\frac{3}{4 \pi G \rho}} = \SI{1.2e+13}{\second} \\
    t_c &= \frac{\rho_p r}{3 Z \rho} \sqrt{\frac{\mu m_{\ce{H}}}{3kT}} \\
    &= \frac{\rho_p r}{3 Z \rho} \sqrt{\frac{\cancel{\mu m_{\ce{H}}}}{3\cancel{k}\frac{4 \pi G \cancel{\mu m_{\ce{H}}} \rho R^2}{15\cancel{k}}}} \\
    &= \frac{\rho_p r}{3 Z \rho R} \sqrt{\frac{5}{4 \pi G \rho}} = \SI{6.93e+11}{\second}
\end{align}
so in this case, \(t_c < t_{\text{ff}}\).

\subsection{}

Taking the ratio \(\frac{t_c}{t_{\text{ff}}}\),
\begin{align}
    \frac{t_c}{t_{\text{ff}}} &= \frac{\frac{\pi \rho_p R}{3 \rho_d} \sqrt{\frac{\mu m_{\ce{H}}}{3kT}}}{\sqrt{\frac{3}{4 \pi G \rho}}} \\
    &= \frac{\pi \rho_p r}{3 Z \rho} \sqrt{\frac{4 \pi G \rho \mu m_{\ce{H}}}{9kT}} \\
    &= \left(\frac{\pi \rho_p}{3 Z} \sqrt{\frac{4 \pi G \mu m_{\ce{H}}}{9k}}\right) \frac{r}{\sqrt{T \rho}} \propto \frac{r}{\sqrt{T \rho}}
\end{align}
Since the cloud initially satisifies the inequality, we can treat the collapse as isothermal, so the ratio is only dependent on \(r\) and \(\rho\).
During the process of collapse, both of these terms serve to make \(\frac{t_c}{t_{\text{ff}}}\) smaller, i.e. \(r\) decreases and \(\rho\) increases.
This means that the collapse will remain Jeans-unstable, making star formation likely throughout, and promising.

\end{document}
