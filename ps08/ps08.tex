\documentclass{article}
\usepackage{eecstex}
\usepackage{mhchem}

\newcommand{\water}{\ce{H2O}}
\DeclareSIUnit\year{yr}

\title{ASTRO 7A PS 08}
\author{Bryan Ngo}
\date{2020-11-08}

\begin{document}

\maketitle

\section{Mean Molecular Weight}

\subsection{}

Using the ionized version for mean molecular weight
\begin{equation}
    \frac{1}{\mu_i} = 2X + \frac{3}{4}Y + \frac{1}{2}Z
\end{equation}
The final composition of the star is currently \(30\% \cdot 81\% = 24.3\%\) hydrogen, \((70\% \cdot 81\%) + 18\% = 74.7\%\) helium, and \(1\%\) other metals, meaning that \(X = 0.243\), \(Y = 0.747\), and \(Z = 0.01\), so \(\mu_i = \num{0.955}\).

\subsection{}

For a neutral composition,
\begin{equation}
    \frac{1}{\mu_n} = X + \frac{1}{4}Y + \left\langle\frac{1}{A}\right\rangle_n Z
\end{equation}
where our \(X, Y, Z\) are identical and \(\left\langle\frac{1}{A}\right\rangle_n = \frac{1}{15.5}\), meaning that \(\mu_n = \num{2.33}\).
This means that \(\frac{\mu_n}{\mu_i} = \num{2.43}\), so an ionized gas has a lower molecular weight, probably due to all the lost electrons due to ionization.

\section{Solar Fusion}

\begin{align}
    \epsilon_{pp} &= \epsilon_{0, pp} \rho X^2 T_6^4 \\
    \epsilon_{CNO} &= \epsilon_{0, CNO} \rho X X_{CNO} T_6^{19.9}
\end{align}

\subsection{}

\begin{equation}
    \frac{\epsilon_{pp}}{\epsilon_{CNO}} = \frac{\epsilon_{0, pp} \cancel{\rho} X^{\cancel{2}} T_6^4}{\epsilon_{0, CNO} \cancel{\rho} \cancel{X} X_{CNO} T_6^{19.9}} = \frac{\epsilon_{0, pp} X}{\epsilon_{0, CNO} X_{CNO} T_6^{15.9}}
\end{equation}
Plugging in \(\epsilon_{0, pp} = \SI{1.08e-12}{\watt\meter\cubed\per\kilogram\squared}\), \(\epsilon_{0, CNO} = \SI{8.24e-31}{\watt\meter\cubed\per\kilogram\squared}\), \(T_6 = \num{15.696}\), \(X = \num{0.3397}\), and \(X_{CNO} = \num{0.0141}\), we get that \(\frac{\epsilon_{pp}}{\epsilon_{CNO}} = \num{3.06}\).

\subsection{}

Taking our previous expression and solving for \(\frac{\epsilon_{pp}}{\epsilon_{CNO}} = \frac{1}{2}\),
\begin{equation}
    \frac{\epsilon_{0, pp} X}{\epsilon_{0, CNO} X_{CNO} T_6^{15.9}} = \frac{1}{2} \implies T_6 = \left(\frac{2 \epsilon_{0, pp} X}{\epsilon_{0, CNO} X_{CNO}}\right)^{\frac{1}{15.9}} = \num{17.6} \implies T = \SI{1.76e+7}{\kelvin}
\end{equation}

\section{Stellar Lifetimes}

\begin{align}
    M_A &= \num{0.072} M_\odot \\
    \log(T_{e, A}) &= \num{3.23} \\
    \log\left(\frac{L_A}{L_\odot}\right) &= \num{-4.3} \\
    M_B &= \num{85} M_\odot \\
    \log(T_{e, B}) &= \num{4.075} \\
    \log\left(\frac{L_B}{L_\odot}\right) &= \num{6.006}
\end{align}

\subsection{}

Using the nuclear timescale equation and given that \num{100}\% of the hydrogen is burned,
\begin{align}
    E_A &= 1 \cdot \num{0.007} M_A c^2 = \SI{9.06e+43}{\joule} \\
    L_A &= L_\odot 10^{\num{-4.3}} = \SI{1.92e+22}{\watt} \\
    t_A &= \frac{E_A}{L_A} = \SI{4.72e+21}{\second} = \SI{1.50e+14}{\year} \\
    E_B &= 1 \cdot \num{0.007} M_B c^2 = \SI{1.06e+47}{\joule} \\
    L_B &= L_\odot 10^{\num{6.006}} = \SI{3.88e+32}{\watt} \\
    t_B &= \frac{E_B}{L_B} = \SI{2.75e+14}{\second} = \SI{8.73e+6}{\year} \\
    \frac{t_A}{t_B} &= \num{1.71e+7}
\end{align}
Star A will last about \num{10000} times longer than the Sun, while Star B will last about \num{10000} times shorter than the Sun.

\subsection{}

Using the Stefan-Boltzmann equation,
\begin{align}
    T_{e, A} &= 10^{\num{3.23}} = \SI{1698}{\kelvin} \\
    R_A &= \sqrt{\frac{L_A}{4 \pi \sigma T_{e, A}^4}} = \SI{5.69e+7}{\meter} = \num{0.08} R_\odot \\
    T_{e, B} &= 10^{\num{4.705}} = \SI{50699}{\kelvin} \\
    R_B &= \sqrt{\frac{L_B}{4 \pi \sigma T_{e, B}^4}} = \SI{9.08e+9}{\meter} = \num{13} R_\odot \\
    \frac{R_B}{R_A} &= \num{159.6}
\end{align}

\section{Hydrostatic Equilibrium}

\subsection{}

Given the equation
\begin{equation}
    \frac{dP}{dr} = -\rho_{\water} g
\end{equation}
we can integrate with respect to a distance \(r\) to obtain
\begin{equation}
    P(z) = \int_z^0 -\rho_{\water} g \, dr = \rho_{\water} g z
\end{equation}

\subsection{}

Examining the forces acting on the block of soap,
\begin{equation}
    F_{net} = F_g - F_p = 0 \implies F_g = F_P
\end{equation}
Subtituting relevant variables,
\begin{equation}
    \rho_{soap} h A g = \rho_{\water} z A g
\end{equation}
where \(A\), \(h\), and \(z\) is the cross-sectional area of the soap, thickness of the soap, and the immersion depth of the soap.
Then, 
\begin{align}
    \rho_{soap} h A \cancel{g} &= \rho_{\water} z A \cancel{g} \\
    m_{soap} &= m_{\water}
\end{align}
meaning the mass of the soap is equal to the displaced mass of water.

\subsection{}

The block of soap should rise higher since its density is the lowest out of all three items.
The ratio of their displacement heights is
\begin{equation}
    \frac{z_w}{z_s} = \frac{\cancel{h} \frac{\rho_w}{\cancel{\rho_{\water}}}}{\cancel{h} \frac{\rho_s}{\cancel{\rho_{\water}}}} = \frac{\rho_w}{\rho_s} > 0
\end{equation}
which, according to our coordinate system, means the wood sinks more.

\section{Optical Depth of Earth's Atmosphere}

\subsection{}

Assuming that the Earth's atmosphere is neutral,
\begin{equation}
    \frac{1}{\mu} = \sum_j \frac{X_j}{A_j} = \frac{m_{\ce{H}}}{m_{\ce{N2}}} X_{\ce{N2}} + \frac{m_{\ce{H}}}{m_{\ce{O2}}} X_{\ce{O2}}
\end{equation}
Letting \(m_{\ce{H}} = \SI{1.00}{\atomicmassunit}\), \(m_{\ce{N2}} = \SI{28.00}{\atomicmassunit}\), \(m_{\ce{O2}} = \SI{32.00}{\atomicmassunit}\), \(X_{\ce{N2}} = \num{0.8}\), and \(X_{\ce{O2}} = \num{0.2}\), \(\mu = \num{28.72}\).

\subsection{}

Given that the mean free path of a photon and the equation for optical depth are, respectively,
\begin{align}
    \tau &= \int_0^\infty \kappa_\lambda \rho \, ds = \int_0^\infty n \sigma_\lambda \, ds \\
    \sigma &= 0.8 \sigma_{\ce{N}} + 0.2 \sigma_{\ce{O}} \\
    n &= \frac{P(s)}{kT}
\end{align}
where we take a weighted sum of the cross-sectional areas of nitrogen and oxygen and derive the number density from the ideal gas law,
\begin{equation}
    \tau = \int_0^\infty \frac{P(s)}{kT} (0.8 \sigma_{\ce{N}} + 0.2 \sigma_{\ce{O}}) \, ds
\end{equation}
Then, assuming hydrostatic equilibrium, we can solve the equation
\begin{align}
    \diff{r} P &= -\rho g \\
    P &= \frac{\rho}{\mu m_{\ce{H}}} kT \\
    \Rightarrow \diff{r} P &= -\frac{\mu m_{\ce{H}} G M_\oplus}{kT R_\oplus^2} P \\
    \Rightarrow P(r) &= P_0 \exp\left(-\frac{\mu m_{\ce{H}} G M_\oplus}{kT R_\oplus^2} r\right) \\
    \Rightarrow \tau &= \int_0^\infty \frac{1}{kT} P_0 \exp\left(-\frac{\mu m_{\ce{H}} G M_\oplus}{kT R_\oplus^2} r\right) (0.8 \sigma_{\ce{N}} + 0.2 \sigma_{\ce{O}}) \, ds \\
    &= -\frac{0.8\sigma_{\ce{N}} + 0.2\sigma_{\ce{O}}}{\cancel{kT}} \left(\frac{\cancel{kT} P_0 R_\oplus^2}{\mu m_{\ce{H}} G M_\oplus}\right) \left.\exp\left(-\frac{\mu m_{\ce{H}} G M_\oplus}{kT R_\oplus^2} r\right) \right|_0^\infty \\
    &= -\frac{(0.8\sigma_{\ce{N}} + 0.2\sigma_{\ce{O}}) P_0 R_\oplus^2}{\mu m_{\ce{H}} G M_\oplus} \left(\cancelto{0}{\lim_{R \to \infty} \exp\left(-\frac{\mu m_{\ce{H}} G M_\oplus}{kT R_\oplus^2} R\right)} - 1\right) \\
    &= \frac{(0.8\sigma_{\ce{N}} + 0.2\sigma_{\ce{O}}) P_0 R_\oplus^2}{\mu m_{\ce{H}} G M_\oplus}
\end{align}

\subsection{}

Given
\begin{align}
    \mu &= \num{28.72} \\
    m_{\ce{H}} &= \SI{1.67e-27}{\kilogram} \\
    P_0 &= \SI{101325}{\pascal} \\
    R_\oplus &= \SI{6.4e+6}{\meter} \\
    M_\oplus &= \SI{6.0e+24}{\meter} \\
    \sigma_{\ce{N}} &= \SI{e-27}{\centi\meter\squared} = \SI{e-31}{\meter\squared} \\
    \sigma_{\ce{O}} &= \SI{2e-27}{\centi\meter\squared} = \SI{2e-31}{\meter\squared} \\
    \tau &= \frac{(0.8\sigma_{\ce{N}} + 0.2\sigma_{\ce{O}}) P_0 R_\oplus^2}{\mu m_{\ce{H}} G M_\oplus} = \num{0.026}
\end{align}

\end{document}
